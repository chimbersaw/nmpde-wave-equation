\documentclass[11pt,a4paper]{article}
\usepackage[margin=1in]{geometry}
\usepackage{amsmath,amssymb,amsthm}
\usepackage{booktabs}
\usepackage{graphicx}
\usepackage{hyperref}
\usepackage{enumitem}
\usepackage{array}

\title{Finite Element Solution of the 2D Wave Equation\\
\large Fully Distributed MPI Implementation with $\theta$-Family and Newmark-$\beta$}
\author{NM-PDE Project}
\date{\today}

\begin{document}
\maketitle

\section{Introduction and Problem Statement}
We solve the wave-equation problem: find $u(x,t)$ such that
\begin{equation}
  u_{tt} - c^2 \Delta u = f \quad \text{in } \Omega \times (0,T],
\end{equation}
with initial conditions
\begin{equation}
  u(\cdot,0)=u_0,\qquad u_t(\cdot,0)=u_1,
\end{equation}
and Dirichlet boundary data on $\partial\Omega$.

The implementation is a config-driven MPI C++ solver using deal.II and Trilinos, with simplicial finite elements on triangular Gmsh meshes, two time-discretization families ($\theta$ and Newmark-$\beta$), ParaView output, and automated convergence studies.

\section{Continuous Model and Weak Formulation}
\subsection{Strong Form and Functional Setting}
Let $\Omega\subset\mathbb{R}^2$ be bounded. The weak formulation is posed in Sobolev spaces $H^1(\Omega)$ and $H_0^1(\Omega)$ \cite{salsa}. For homogeneous Dirichlet data, test functions belong to $V=H_0^1(\Omega)$.

\subsection{Homogeneous Dirichlet Weak Form and Semi-Discrete System}
For $v\in V$, multiplying by $v$ and integrating by parts gives
\begin{equation}
  \langle u_{tt},v\rangle + c^2(\nabla u,\nabla v)=(f,v),\qquad \forall v\in V.
\end{equation}
Using a simplicial FE subspace $V_h=\text{span}\{\varphi_i\}_{i=1}^N$ and
$u_h(x,t)=\sum_{j=1}^N U_j(t)\varphi_j(x)$, one obtains the matrix ODE
\begin{equation}
  M\ddot{\mathbf U}(t) + c^2 K\mathbf U(t)=\mathbf F(t),
\end{equation}
with
\begin{equation}
  M_{ij}=\int_\Omega \varphi_i\varphi_j\,dx,\quad
  K_{ij}=\int_\Omega \nabla\varphi_i\cdot\nabla\varphi_j\,dx,\quad
  F_i(t)=\int_\Omega f(\cdot,t)\varphi_i\,dx.
\end{equation}

\subsection{Non-Homogeneous Dirichlet Data via Lifting}
Following the standard lifting argument \cite{quarteroni,salsa}, for boundary data $u=g$ on $\partial\Omega$, write
\begin{equation}
  u = w + \widetilde g, \qquad w\in H_0^1(\Omega), \qquad \widetilde g|_{\partial\Omega}=g.
\end{equation}
Then $w$ solves
\begin{equation}
  \langle w_{tt},v\rangle + c^2(\nabla w,\nabla v)
  = (f,v) - \langle \widetilde g_{tt},v\rangle - c^2(\nabla\widetilde g,\nabla v).
\end{equation}
In matrix form:
\begin{equation}
  M\ddot{\mathbf W}+c^2K\mathbf W = \mathbf F - M\ddot{\mathbf G} - c^2K\mathbf G.
\end{equation}
In the code, Dirichlet values are imposed strongly at each step on all boundary IDs
via constraints and boundary-value elimination. This is algebraically equivalent to
working with lifted unknowns and interior DoFs.

\section{Space Discretization and Fully Distributed MPI Implementation}
The space discretization follows the project requirements and simplicial-FEM theory \cite{quarteroni}.

\subsection{Mesh and FE Space}
\begin{itemize}[leftmargin=*]
  \item Mesh input: Gmsh triangle meshes (\texttt{.msh}) generated from \texttt{mesh/*.geo}.
  \item Triangulation: \texttt{parallel::fullydistributed::Triangulation<2>}.
  \item FE: runtime-selectable \texttt{FE\_SimplexP<2>(degree)} (default degree 1).
  \item Quadrature: only \texttt{QGaussSimplex<2>}.
  \item DoFs/constraints: \texttt{DoFHandler<2>} + hanging-node constraints + Dirichlet BC on all boundary IDs with configured boundary function values.
\end{itemize}

\subsection{Distributed Mesh Import Pipeline}
The code uses this pipeline:
\begin{enumerate}[leftmargin=*]
  \item Read Gmsh mesh into serial \texttt{Triangulation<2>} with \texttt{GridIn}.
  \item Partition with \texttt{GridTools::partition\_triangulation}.
  \item Build a \texttt{TriangulationDescription}.
  \item Create the fully distributed triangulation from the description.
\end{enumerate}

\subsection{Assembly and Linear Algebra}
Mass and stiffness matrices are assembled once. The forcing vector $\mathbf F(t)$ is assembled each step.
All vectors/matrices are distributed Trilinos MPI objects. Implicit systems are solved with Krylov iterations (CG) and AMG-style preconditioning (Trilinos \texttt{PreconditionAMG}).

\section{Time Discretization}
The report peer examples (\texttt{2.pdf}, \texttt{5.pdf}) emphasized explicit update formulas and stability/dispersion implications. The present report includes those details for both families.

\subsection{$\theta$-Method Family}
Introduce $\mathbf v=\dot{\mathbf u}$, giving the first-order system
\begin{equation}
\dot{\mathbf u}=\mathbf v,
\qquad
M\dot{\mathbf v}+c^2K\mathbf u=\mathbf F(t).
\end{equation}
With step $\Delta t$ and weight $\theta\in[0,1]$:
\begin{align}
  \frac{\mathbf u^{n+1}-\mathbf u^n}{\Delta t}
  &= \theta\mathbf v^{n+1}+(1-\theta)\mathbf v^n,\\
  M\frac{\mathbf v^{n+1}-\mathbf v^n}{\Delta t}
  &+c^2K\left(\theta\mathbf u^{n+1}+(1-\theta)\mathbf u^n\right)=\mathbf F_\theta^{n+1},
\end{align}
where $\mathbf F_\theta^{n+1}=\theta\mathbf F^{n+1}+(1-\theta)\mathbf F^n$.
For $\theta>0$, eliminating $\mathbf v^{n+1}$ gives an SPD displacement solve with effective matrix
\begin{equation}
  A_\theta = \frac{1}{\Delta t^2}M + c^2\theta^2K.
\end{equation}
Special cases:
\begin{itemize}[leftmargin=*]
  \item $\theta=0$: forward Euler-type explicit update (conditionally stable).
  \item $\theta=0.5$: Crank--Nicolson (second-order, low dissipation, phase error dominates when $\Delta t$ is large).
  \item $\theta=1$: backward Euler (first-order, strongly dissipative).
\end{itemize}

\subsection{Newmark-$\beta$ Family}
The Newmark update acts directly on
$M\ddot{\mathbf u}+c^2K\mathbf u=\mathbf F$:
\begin{align}
  \mathbf u^{n+1} &= \mathbf u^n + \Delta t\mathbf v^n
   + \Delta t^2\left(\frac12-\beta\right)\mathbf a^n + \beta\Delta t^2\mathbf a^{n+1},\\
  \mathbf v^{n+1} &= \mathbf v^n + \Delta t(1-\gamma)\mathbf a^n + \Delta t\gamma\mathbf a^{n+1}.
\end{align}

For $\beta>0$, the implementation solves for $\mathbf u^{n+1}$ through
\begin{equation}
  \left(\frac{1}{\beta\Delta t^2}M + c^2K\right)\mathbf u^{n+1}=\mathbf r^{n+1},
\end{equation}
then reconstructs $\mathbf a^{n+1}$ and $\mathbf v^{n+1}$. For $\beta=0$ (central-difference/leapfrog setting with $\gamma=0.5$), the code uses explicit displacement prediction and a mass solve for acceleration.

Important presets:
\begin{itemize}[leftmargin=*]
  \item $(\beta,\gamma)=(0.25,0.5)$ average-acceleration: second-order, unconditionally stable, minimal algorithmic dissipation.
  \item $(\beta,\gamma)=(0,0.5)$ central difference style: explicit/conditionally stable, CFL-restricted.
\end{itemize}

\subsection{Dissipation and Dispersion (Concise Discussion)}
Consistent with standard references \cite{quarteroni,salsa} and peer-report observations:
\begin{itemize}[leftmargin=*]
  \item Strongly implicit first-order damping schemes (e.g., backward Euler) introduce noticeable numerical dissipation.
  \item Crank--Nicolson and average-acceleration Newmark preserve amplitudes better but can show phase drift (dispersion) when $\Delta t$ is not small enough.
  \item Explicit schemes can be low-dissipative in stable CFL regimes, but instability appears quickly once CFL is violated.
\end{itemize}

\section{Software Architecture and Config Workflow}
\subsection{Source Organization}
\begin{itemize}[leftmargin=*]
  \item \texttt{src/wave\_solver.*}: FE setup, assembly, MPI vectors/matrices, output, solver interface.
  \item \texttt{src/integrators/theta\_integrator.cpp} + \texttt{src/theta\_integrator.hpp}.
  \item \texttt{src/integrators/newmark\_integrator.cpp} + \texttt{src/newmark\_integrator.hpp}.
  \item \texttt{src/config.*}: strict key=value parser with method-aware validation.
  \item \texttt{src/convergence\_runner.*}: spatial/temporal sweep driver and CSV writer.
  \item \texttt{src/function\_factory.*}: manufactured solutions / forcing / boundary scenarios.
\end{itemize}

\subsection{CLI and Config}
CLI is intentionally minimal:
\begin{verbatim}
wave-equation --config path/to/file.cfg
\end{verbatim}

Key config blocks:
\begin{itemize}[leftmargin=*]
  \item Core run:
  \texttt{mode, method, mesh\_file, fe\_degree, wave\_speed, dt, n\_steps}.
  \item Output/scenario:
  \texttt{output\_interval, output\_dir, scenario\_u0, scenario\_u1,}
  \texttt{scenario\_f, scenario\_bc}.
  \item Method-specific:
  \texttt{theta} for the $\theta$ family;
  \texttt{beta} and \texttt{gamma} for Newmark.
  \item Convergence:
  \texttt{convergence\_mesh\_files},
  \texttt{convergence\_dt\_values},
  \texttt{convergence\_csv\_space},
  \texttt{convergence\_csv\_time}.
\end{itemize}

Unknown keys and invalid ranges are rejected during parsing.

\section{Convergence Methodology and Results}
\subsection{Manufactured Reference and Error Norms}
To measure error, we use the standing-wave exact solution on $\Omega=[0,1]^2$:
\begin{equation}
  u_{\text{exact}}(x,y,t)=\cos\!\left(c\sqrt{2}\pi t\right)\sin(\pi x)\sin(\pi y),
\end{equation}
with compatible initial/boundary data and zero forcing.

At final time $t_{\mathrm f}$, errors are computed in:
\begin{itemize}[leftmargin=*]
  \item $L^2$ norm,
  \item full $H^1$ norm via \texttt{VectorTools::H1\_norm} (as requested).
\end{itemize}

\subsection{Automated Studies and CSV Output}
Implemented modes:
\begin{itemize}[leftmargin=*]
  \item \texttt{convergence\_space}: vary mesh hierarchy, fixed $\Delta t$.
  \item \texttt{convergence\_time}: vary $\Delta t$, fixed fine mesh.
  \item \texttt{convergence\_both}: run both and write two CSV files.
\end{itemize}

Each CSV row stores:
\begin{verbatim}
study,method,fe_degree,theta,beta,gamma,mesh_file,dt,n_steps,t_final,h,ndofs,
l2_error,h1_error,observed_order_l2,observed_order_h1
\end{verbatim}

\subsection{Observed Behavior from Current Runs}
A representative run with implicit Newmark ($\beta=0.25,\gamma=0.5$), degree-1 FE, and generated unit-square mesh hierarchy gives:
\begin{itemize}[leftmargin=*]
  \item Spatial study: observed $L^2$ orders around $\approx 1.7$--$2.6$ and $H^1$ around $\approx 0.87$--$1.35$, consistent with expected $\mathcal O(h^2)$ and $\mathcal O(h)$ asymptotics for $P1$.
  \item Temporal study on a fixed mesh: error is nearly saturated (very small slope), indicating spatial error dominates for the selected mesh/time range.
\end{itemize}

Figure~\ref{fig:conv} is produced by \texttt{scripts/plot\_convergence.py} from the generated CSV files.

\begin{figure}[h]
  \centering
  \includegraphics[width=0.82\textwidth]{../results/convergence.png}
  \caption{Log--log convergence plot automatically generated from CSV outputs.}
  \label{fig:conv}
\end{figure}

\section{How to Build and Run}
\subsection{Build}
\begin{verbatim}
cmake -S . -B cmake-build-release -DCMAKE_BUILD_TYPE=Release
cmake --build cmake-build-release -j
\end{verbatim}

\subsection{Mesh Generation}
\begin{verbatim}
./scripts/generate_mesh.sh
\end{verbatim}

\subsection{Solve Mode}
\begin{verbatim}
mpirun -n 4 ./cmake-build-release/wave-equation \
  --config configs/standing_wave/theta_crank_nicolson.cfg
\end{verbatim}

\subsection{Convergence + Plot}
\begin{verbatim}
mpirun -n 4 ./cmake-build-release/wave-equation \
  --config configs/convergence/convergence_both.cfg
python3 scripts/plot_convergence.py \
  --space-csv results/convergence_space.csv \
  --time-csv results/convergence_time.csv \
  --output results/convergence.png
\end{verbatim}

ParaView files (\texttt{.pvtu/.vtu}) are written to \texttt{solution/}.

\section{Conclusions}
The project now provides a fully distributed simplicial-FEM wave solver aligned with the requested workflow: separate $\theta$ and Newmark integrators, strict config-driven execution, AMG-preconditioned Krylov implicit solves, ParaView export, and automated convergence CSV/plot generation. The report includes full weak-form context, method formulas, and implementation-to-theory mapping.

\bibliographystyle{plain}
\bibliography{references}

\end{document}
